%
%  Social Hosting
%
%  Created by Samuel Esposito and Grayson Kleine on 2011-04-15.
%
\documentclass[a4paper, 10pt]{article}

\usepackage[english]{babel}

% Use utf-8 encoding for foreign characters
\usepackage[utf8]{inputenc}

% Setup for fullpage use
\usepackage{fullpage}

\usepackage{xifthen}

\usepackage{float}

% Uncomment some of the following if you use the features
%
% Running Headers and footers
%\usepackage{fancyhdr}

% Multipart figures
%\usepackage{subfigure}

% More symbols
%\usepackage{amsmath}
%\usepackage{amssymb}
%\usepackage{latexsym}

% Surround parts of graphics with box
\usepackage{boxedminipage}

% Package for including code in the document
\usepackage{listings}
\usepackage[table]{xcolor}
\usepackage{color}

\usepackage{amsmath}

% If you want to generate a toc for each chapter (use with book)
\usepackage{minitoc}

% This is now the recommended way for checking for PDFLaTeX:
\usepackage{ifpdf}

%\newif\ifpdf
%\ifx\pdfoutput\undefined
%\pdffalse % we are not running PDFLaTeX
%\else
%\pdfoutput=1 % we are running PDFLaTeX
%\pdftrue
%\fi

\ifpdf
\usepackage[pdftex]{graphicx}
\else
\usepackage{graphicx}
\fi

\newcommand{\image}[3][\textwidth]{%
  \begin{figure}[H]%
    \centering%
    \setlength\fboxsep{0pt}\fbox{\includegraphics[width={#1}]{#2}}%
    \ifthenelse{\isempty{#3}}%
      {}%
      {\caption{{#3}}}%
    \label{fig:{#2}}%
  \end{figure}
}

\renewcommand{\familydefault}{\sfdefault}

\title{Fall Detection on Android\\~\\Ubiquitous Computing -- 2010-2011\\~}
\author{Samuel Esposito \& Grayson Kleine}

\date{}

\definecolor{grijs}{rgb}{.92,.92,.92}
\lstset{
  language=java,
  basicstyle=\footnotesize,
  showstringspaces=false,
  numbers=left,
  numberstyle=\footnotesize,
  stepnumber=1,
  numbersep=5pt,
  backgroundcolor=\color{grijs},
  showspaces=false,
  showtabs=false
  commentstyle=\itshape,
  tabsize=8,
  postbreak=,
  breaklines=true
}

\begin{document}

\ifpdf
\DeclareGraphicsExtensions{.pdf, .jpg, .tif}
\else
\DeclareGraphicsExtensions{.eps, .jpg}
\fi

\maketitle

\section{Introduction}
In this document we discuss the Android based fall detection software we built for the Ubiquitous Computing course at the University of Groningen. The intent of the project was to create an Android application that would detect a person's fall while carrying an Android smartphone. The application should cleverly process the smartphone's sensor data in order to distinguish a dangerous fall from other daily activities that involve sudden movements, such as sitting down or running. Unless the user cancels the application within 10 seconds after a fall was detected, it should send out a call for aid.

In the next section we briefly discuss a set of algorithms for the calculation of fall features that are indispensable to sensitive and specific fall detection. After that we discuss our implementation of the fall detection software on Android and document the steps to take in order to deploy it, including simulation. Finally we evaluate the project and discuss our main results.

\section{Fall Features}
There is plenty of literature describing fall features which can be calculated from raw sensor data. We chose to use the article on fall detection of A. K. Bourke et. \textit{al.}\footnote{Assessment of waist-worn tri-axial accelerometer based fall-detection algorithms using continuous unsupervised activities. Bourke AK, van de Ven P, Gamble M, O'Connor R, Murphy K, Bogan E, McQuade E, Finucane P, Olaighin G, Nelson J. Conf Proc IEEE Eng Med Biol Soc. 2010;2010:2782-5. PMID: 21095967 [PubMed - indexed for MEDLINE]} as a base for this project because it gives an up to date overview of the state of the art and compares different algorithms for sensitivity (detecting a fall when it occurs) and specificity (discerning a fall from other activities). The results of their research is that the combination of three features yields the best results in terms of sensitivity and specificity: vertical velocity, fall impact and posture. We discuss each of these features separately below.

The fall impact is the most commonly used fall feature as it is most easy to derive from the raw data of a tri-axial accelerometer. It is calculated as the root sum of squares of the acceleration on each axis, from now on referred to as $RSS$: \[RSS = \sqrt{x^2 + y^2 + z^2}\] The $RSS$ gives an indication of the forces that act on the smartphone during impact measured in $g$ (gravity). The harder a person falls, the bigger the $g$-forces on the smartphone will be and the higher the $RSS$ value is. Based on their findings and the consensus in the field of fall detection, Bourke et. \textit{al.} put forward a threshold of $2.8g$ for the $RSS$ value: every value higher than this threshold is considered as a fall feature.

Another less straightforward feature of a fall is the vertical velocity. When a person makes a free fall, the g-force on the smartphone is compensated by the downward acceleration for a small period of time: the smartphone experiences zero-gravity. This fall feature, henceforth referred to as $VVE$ (vertical velocity), is calculated by numerically integrating the difference of the $RSS$ with $1g$ (9.80665) for a small time window ($\Delta T = 0.6s$): \[VVE = \int\limits_{\Delta T} \! (RSS(t) - 1g) \, \mathrm{d}t\] The higher the vertical velocity of the smartphone, the lower the value of the $VVE$ feature. Bourke et. \textit{al.} determined that a fall should be detected when the value of $VVE$ drops below $-0.7$.

Finally there is the human posture to take into account. When a person makes a dangerous fall for which aid is needed, he or she is expected to remain in a horizontal position for some time after the fall. When the smartphone is carried in a pocket, it should also maintain this horizontal position. A horizontal position in the context of fall detection is defined as follows: the smartphone must have an angle between 0º and 30º from the horizon for 75\% of the time during a 2 second interval 1 second after an $RSS$ or $VVE$ feature was detected.

As mentioned above, Bourke et. \textit{al.} showed in their article that the best results are obtained when the combination of vertical velocity, fall impact and posture is used for detecting a fall. More specifically the detection of a fall should be marked by either a $VVE$ feature followed by a posture feature or an $RSS$ feature followed by a posture feature.

Now that the theoretical details behind fall detection are clear we will shift focus to the implementation of the above algorithms on Android.

\section{Fall Detection on Android}
In this section we discuss the implementation of the fall detection application for the Android platform. First we explain how to retrieve sensor data on an Android smartphone and implement the algorithms for calculating the different fall features\footnote{$VVE$ (vertical velocity), $RSS$ (fall impact) and posture (See chapter 2: Fall Features).} from this raw data. Next we handle sending a fall notification to a RESTfull web service and allowing the user to cancel the sending within 10 seconds. Finally we wrap this all together into an Android activity.

\subsection{Calculating Fall Features}
To get access to sensor data on the Android platform one has to create a class implementing the `SensorEventListener' interface and register it with the android `SensorManager' as shown in the listing below.
\begin{lstlisting}
public class FallDetector implements
    SensorEventListener {
    
  private SensorManager mSensorManager = 
    (SensorManager) activity.getSystemService(activity.SENSOR_SERVICE);
    
  public void registerListeners() {
    mSensorManager.registerListener(this,
        mSensorManager.getDefaultSensor(Sensor.TYPE_ACCELEROMETER),
        SensorManager.SENSOR_DELAY_UI);
    mSensorManager.registerListener(this,
        mSensorManager.getDefaultSensor(Sensor.TYPE_ORIENTATION),
        SensorManager.SENSOR_DELAY_UI);
  }  
  ...  
}
\end{lstlisting}
\newpage
By implementing the `onSensorChanged' method the sensor data can be captured and processed. We are particularly interested in the accelerometer data for calculating the $RSS$ and $VVE$ features and the orientation sensor for deriving the posture feature. The listing below shows how to filter for the right type of sensor and access the available data.
\begin{lstlisting}
@Override
public void onSensorChanged(SensorEvent event) {
  synchronized (this) {
    if (event.sensor.getType() == Sensor.TYPE_ACCELEROMETER) {
      event.values[0]; // Acceleration minus Gx on the x-axis
      event.values[1]; // Acceleration minus Gy on the y-axis
      event.values[2]; // Acceleration minus Gz on the z-axis
    } else if (event.sensor.getType() == Sensor.TYPE_ORIENTATION) {      
      event.values[0]; // Azimuth, angle between the magnetic north direction and the
                       // y-axis, around the z-axis (0 to 359). 
                       // 0=North, 90=East, 180=South, 270=West
      event.values[1]; // Pitch, rotation around x-axis (-180 to 180)
      event.values[2]; // Roll, rotation around y-axis (-90 to 90)
    }
  }
}
\end{lstlisting}
The $RSS$ feature is calculated by taking the root sum of squares of the accelerometer event values. When the values exceeds the threshold ($2.8g$), an $RSS$ feature is detected. This is implemented below.
\begin{lstlisting}
protected final float RssTreshold = 2.8f;
...
float rss = (float) Math.sqrt(Math.pow(event.values[0], 2)
                            + Math.pow(event.values[1], 2)
                            + Math.pow(event.values[2], 2));
if (rss > RssTreshold * SensorManager.STANDARD_GRAVITY) {  
  // Rss feature detected!
}
\end{lstlisting}
The $VVE$ feature in turn is calculated by taking the numerical integer\footnote{The numerical integer is approximated using Simpson's rule ($2^{nd}$ order polynomial).\\Source: http://en.wikipedia.org/wiki/Simpson\%27s\_rule} of the $RSS$ minus $1g$ in a time window of $0.6s$. When this value goes below the $VVE$ threshold ($-0.7g$), a $VVE$ feature is detected. This is implemented in the listing below.
\begin{lstlisting}
private float mRssValues[] = new float[256];
private int mRssCount = 0;
private int mRssIndex = 0;
private long RssStartTime = 0;
protected final float VveWindow = 0.6f;
protected final float VveTreshold = -0.7f;
...
// Store all RSS values in the window in a circular array
if (RssStartTime == 0) {
  RssStartTime = date.getTime();
  mRssCount++;
} else if (date.getTime() - RssStartTime <= VveWindow * 1000
    && mRssCount < mRssValues.length) {
  mRssIndex = mRssCount++;
} else {
  mRssIndex = ++mRssIndex % mRssCount;
}
mRssValues[mRssIndex] = rss
    - SensorManager.STANDARD_GRAVITY;
// Calculate the numerical integer over all stored RSS values
float vve = 0;
for (int i = 0; i < mRssCount; i++) {
  vve += mRssValues[i];
}
vve = (vve * VveWindow) / mRssCount;
if (vve < VveTreshold * SensorManager.STANDARD_GRAVITY) {
  // Vve feature detected!
}
\end{lstlisting}
One second after either a $VVE$ feature or an $RSS$ feature was detected, posture data is collected in a 2 second time window in order to detect the posture feature. The posture data itself is derived from the orientation sensor's pitch value. When more than 75\% of the collected posture data in the window has an angle smaller than 30º to the horizon, the posture feature has been detected.
\begin{lstlisting}
private final int OriOffset = 1000;
private final int OriWindow = 2000;
private long OriStartTime = 0;
protected final float OriTreshold = 60;
private final float OriConstraint = 0.75f;
private float OriValues[] = new float[256];
private int ori_index = 0;
...
// Calculate orientation wrt horizon
float ori = (90 - Math.abs(event.values[1]));
// Wait one second
long wait_interval = (activity.RssTime != 0 ? date
    .getTime() - activity.RssTime
    : (activity.VveTime != 0 ? date.getTime()
        - activity.VveTime : 0));
if (wait_interval >= OriOffset) {
  // Collect ori values for 2 seconds
  if (OriStartTime == 0)
    OriStartTime = date.getTime();
  else if (date.getTime() - OriStartTime < OriWindow) {
    if (ori_index < OriValues.length)
      OriValues[ori_index++] = ori;
  } else {
    // Calculate percentage above threshold
    int count = 0;
    for (int i = 0; i < ori_index; i++) {
      if (OriValues[i] > OriTreshold)
        count++;
    }
    if (count / ori_index >= OriConstraint) {
       // Posture feature detected
    }
  }
}
\end{lstlisting}
When either a $VVE$ feature or an $RSS$ feature has been detected followed by the detection of a posture feature, a fall has been detected. The fall is handled by sending a call for aid to a RESTfull web service, which is described in the next section.

\newpage
\subsection{RESTfull Fall Notification}
For every fall that has been detected, a fall notification is sent to a RESTfull web service. Before sending the notification, a progress dialog is shown offering the user the opportunity to cancel it within 10 seconds. With every fall notification the current user location, if available, is included in the message. Below the code for accessing the smartphone's location is listed.
\begin{lstlisting}
public class LocationUpdateHandler implements LocationListener {

  private FallActivity activity;
  protected LocationManager locationManager;

  public LocationUpdateHandler(FallActivity activity) {
    this.activity = activity;
    // Get location manager
    locationManager = (LocationManager) activity
        .getSystemService(Context.LOCATION_SERVICE);
    // Request location updates
    locationManager.requestLocationUpdates(
        LocationManager.GPS_PROVIDER, 0, 0,
        activity.locationUpdateHandler);
  }

  // Handle location updates
  public void onLocationChanged(Location loc) {
    synchronized (this) {
      activity.latitude = loc.getLatitude();
      activity.longintude = loc.getLongitude();
    }
  }
}
\end{lstlisting}
Beneath the actual fall notification is compiled and sent.
\begin{lstlisting}
// Making an HTTP post request and reading out the response
HttpClient httpclient = new DefaultHttpClient();
httpclient.getParams().setParameter(
    CoreConnectionPNames.CONNECTION_TIMEOUT, 10000);
HttpPost httppost = new HttpPost("http://web.service.host/falls");
List<NameValuePair> nameValuePairs = new ArrayList<NameValuePair>(2);
// Set fall timestamp
nameValuePairs.add(new BasicNameValuePair("datetime",
    (activity.VveTime != 0 ? Long.toString(activity.VveTime)
        : (activity.RssTime != 0 ? Long
            .toString(activity.RssTime) : ""))));
// Set RSS feature
nameValuePairs.add(new BasicNameValuePair("rss",
    (activity.RssVal == 0 ? "" : Float.toString(activity.RssVal))));
// Set VVE feature
nameValuePairs.add(new BasicNameValuePair("vve",
    (activity.VveVal == 0 ? "" : Float.toString(activity.VveVal))));
// Set user location
nameValuePairs
    .add(new BasicNameValuePair("lat", Double.toString(latitude)));
nameValuePairs
    .add(new BasicNameValuePair("lon", Double.toString(longitude)));
try {
  httppost.setEntity(new UrlEncodedFormEntity(nameValuePairs));
} catch (UnsupportedEncodingException e) {
  ...
}
// Send the notification
HttpResponse response;
try {
  response = httpclient.execute(httppost);
} catch (Exception e) {
  ...
}
\end{lstlisting}

\subsection{Android Activity}
All of the applications functionality is bundled together in a so called Android activity. In this class the applications view is set and the objects for processing the sensor data are initialized.
\begin{lstlisting}
public class FallActivity extends Activity {

  protected GraphView mGraphView;

  protected FallDetector mFallDetector;

  protected LocationUpdateHandler locationUpdateHandler;

  @Override
  protected void onCreate(Bundle savedInstanceState) {
    // Be sure to call the super class.
    super.onCreate(savedInstanceState);
    // Create the view
    mGraphView = new GraphView(this);
    setContentView(mGraphView);
    // Create the fall detector
    mFallDetector = new FallDetector(this);
    // Initialize location manager
    locationUpdateHandler = new LocationUpdateHandler(this);
    // Check whether gps is turned on
    locationUpdateHandler.checkGPS();
    // Set app orientation to landscape
    setRequestedOrientation(ActivityInfo.SCREEN_ORIENTATION_LANDSCAPE);
  }
}
\end{lstlisting}

\section{Deployment}
If you are to deploy the Android fall detection application on a smartphone or emulator starting from the source code, you need to install an appropriate development environment on your computer. One good option is to go with Eclipse 3.5 or greater\footnote{http://www.eclipse.org/downloads/} including the Java Development Tools plugin. Make sure you have the Java Development Kit 5 or 6 installed. After that you have to download and unpack/install the latest version of the Android SDK Starter Package\footnote{http://developer.android.com/sdk/index.html} and install the Android Development Tools plugin\footnote{http://developer.android.com/sdk/eclipse-adt.html} for Eclipse. 

Now that all necessary software is in place, we will explain how to run the application on the Android emulator. Start by opening the Android SDK manager (from the Eclipse Window menu) and install the Android Platform version 2.1 (the application was not tested on later versions) from the `Android Repository' under the `Available packages' window. Also install the `Android SDK Tools' and `Platform-tools'. Next create a new virtual device that runs on platform version 2.1 in the `Virtual devices' window. Now it is time to add the source code to a fresh Android project in Eclipse. Download the code from https://github.com/krikis/FallDet/zipball/master and extract the zipball in some convenient folder. Next create a new Android project in Eclipse, select the option `Create project from existing source', browse for the previously downloaded source code and click `Finish'. Right click on the project folder and select `Run As $>$ Android Application' to run the fall detection application on the emulator\footnote{Select `Run As $>$ Run Configuration...' and manually select the target to directly run it on your smartphone.}. Yes, it runs in landscape mode: press Ctrl-F11 to turn the screen of your emulator.

When you run the application on the Android emulator, a progress dialog will be displayed waiting for the GPS to be activated. Click `Just go' to skip this step, we will explain it later. Now a graph is shown in which nothing much happens, as there is no sensor data coming in on the Emulator. To get some interesting results we need simulation. For this we use the SensorSimulator from OpenIntents\footnote{http://code.google.com/p/openintents/wiki/SensorSimulator}. This requires a number of steps: open the `platform-tools' folder of the android SDK in a console and type in the following command to install the SensorSimulatorSettings apk on the emulator:
\begin{lstlisting}
$ .adb install /full/path/to/extracted/zipball/doc/SensorSimulatorSettings-1.1.0-rc1.apk
\end{lstlisting}
When the application is successfully installed, start it up on the emulator. Next fire up the SensorSimulator jarfile in the extracted zipball's doc directory (sensorsimulator-1.1.0-rc1.jar). This will open a simulator interface displaying a set of controls for manipulating the sensor data. In a text-area on the left the IP addresses and the port on which the SensorSimulator listens are listed. Enter these details in the corresponding text-fields in the SensorSimulatorSettings app on the emulator, then click the testing tab and click `Connect' (and make sure the emulator's `input method' is set to `Android keyboard' for this). Check all available sensors and see the values change as you fiddle with the controls in the simulator interface.

The connection between the SensorSimulator and the emulator now has been successfully set up. This leaves us only to change the project code to make it get its sensor data from the simulator. First make sure the jar file that resides in the project lib directory (sensorsimulator-lib-1.1.0-rc1.jar) is in the class path: in Eclipse right click the jar file and select `Build Path $>$ Add to Build Path'.  Then open the `FallDetector' java file in the project `src' folder and change it according to the instructions below\footnote{If you are a Git man, just run `git apply enable\_simulator.patch' and you are ready to go.}.

The import lines for the sensor classes
\begin{lstlisting}
//import org.openintents.sensorsimulator.hardware.Sensor;
//import org.openintents.sensorsimulator.hardware.SensorEvent;
//import org.openintents.sensorsimulator.hardware.SensorEventListener;
//import org.openintents.sensorsimulator.hardware.SensorManagerSimulator;

import android.hardware.Sensor;
import android.hardware.SensorEvent;
import android.hardware.SensorEventListener;
import android.hardware.SensorManager;
\end{lstlisting}
should be changed to this
\begin{lstlisting}
import org.openintents.sensorsimulator.hardware.Sensor;
import org.openintents.sensorsimulator.hardware.SensorEvent;
import org.openintents.sensorsimulator.hardware.SensorEventListener;
import org.openintents.sensorsimulator.hardware.SensorManagerSimulator;

// import android.hardware.Sensor;
// import android.hardware.SensorEvent;
// import android.hardware.SensorEventListener;
import android.hardware.SensorManager;
\end{lstlisting}
The declaration of the sensor manager
\begin{lstlisting}
private SensorManager mSensorManager;
\end{lstlisting}
should be this:
\begin{lstlisting}
private SensorManagerSimulator mSensorManager;
\end{lstlisting}
The initialization of the manager
\begin{lstlisting}
mSensorManager = (SensorManager) activity
		.getSystemService(activity.SENSOR_SERVICE);
\end{lstlisting}
has to be changed to this:
\begin{lstlisting}
mSensorManager = SensorManagerSimulator.getSystemService(activity,
    FallActivity.SENSOR_SERVICE);
mSensorManager.connectSimulator();
\end{lstlisting}
Finally the two references to the event sensor type
\begin{lstlisting}
event.sensor.getType() == Sensor.TYPE_ACCELEROMETER
\end{lstlisting}
should look like this:
\begin{lstlisting}
event.type == Sensor.TYPE_ACCELEROMETER
\end{lstlisting}

The application is now ready for interfacing with the SensorSimulator. Fire it up again by right clicking on the project folder and selecting `Run As $>$ Android Application' (or just clicking the play button in the Eclipse toolbar). The application will once more try to activate the GPS. Open a telnet connection to the emulator and type in a geo fix with a longitude and latitude to provide the emulator with location data.
\begin{lstlisting}
$ telnet localhost 5554
Trying 127.0.0.1...
Connected to localhost.
Escape character is '^]'.
Android Console: type 'help' for a list of commands
geo fix 6.5365 53.24015
OK
\end{lstlisting}
The application on the emulator will now happily proceed and a GPS icon appears in the top bar of the interface. Now you will see some lines proceed horizontally through the graph. The top graph displays the vertical velocity of the smartphone ($VVE$). In the middle the fall impact is shown ($RSS$). In the bottom the posture is shown. Switch back to the simulator interface to generate some sensor data (make sure to maximize the window). Select the `move' option and drag the wireframe of the phone around to generate some acceleration data.
\image{simulator.png}{}
To simulate the $RSS$ feature (`Fall Impact' in the graph), just shake the phone's wireframe brusquely or smash it against the sides of your screen. The fall impact line in the graph will peak above the red line of the $RSS$ threshold and a blue mark will indicate the feature was detected. Simulating the $VVE$ feature (`Vertical Velocity' in the graph) is less straightforward as it requires smoothly accelerating the wireframe towards gravity for some period of time ($0.6s$). We were only able to hit a threshold of $-0.3g$ having the wireframe `fall' vertically the length of two adjacent screens. This is naturally because the threshold was selected to detect the fall of a person and not something like a 30cm fall. In order to simulate a complete fall, the posture feature also has to be simulated. This can simply be done by changing the pitch in the simulator interface so that the posture line surfaces the 60º threshold line. 
\newpage
In the image below you see a fall for which all three features were detected. In the uppermost graph the vertical velocity feature is detected where the green line dips below the red threshold line. Shortly after that the fall impact feature is detected in the middle graph when the red line peaks above the threshold line. During the fall the posture in the bottom graph changed from standing to lying. 1 second after the $VVE$ feature was detected the position remains the same for two seconds, so the posture feature is detected.
\image{fall_detected.png}{}
Now a progress dialog is presented to the user to allow him to cancel the sending of a fall notification.
\image{sending_notification.png}{}
\newpage
If the user does not cancel the dialog, a notification is sent to a RESTfull web service. The folder `server\_site' contains a Rails powered web application that receives the notifications and lists them in a web interface for demonstration purposes. 
\image{server_site.png}{}
By clicking on a row in the table the location of the fall is shown on a map with a green arrow.
\image{map.png}{}

\newpage
\section{Evaluation and Future Work}
In this project we built a smartphone application that saves lives. Powered by the latest research on fall detection it detects the fall of a person using three features: the vertical velocity of the smart phone, the fall impact and the posture after the fall. Thanks to the combination of these diverse features the application if very sensitive to a person's fall: it detects falls that have a very prominent vertical velocity, for instance when someone passes out, as well as falls that stand out in fall impact, for instance when falling from the stairs or while cycling. Furthermore because of the broad feature set the application is very specific. The well chosen thresholds for vertical velocity and fall impact allow it to distinguish a fall from other activities involving abrupt movements like sitting down. The posture feature, requiring the person to stay in horizontal position for 2 seconds after the fall, lets the application differ between falling and activities like jumping or running. 

When a fall is detected, a call for aid including the person's location is sent to a service, so that  
the emergency service can easily find the person and offer help in time. If the person does not need any emergency service, he or she can cancel the call for aid within 10 seconds. In the future it would be nice to not only send a message to a service when a fall is detected, but also call an emergency number allowing the person to talk to a nurse in order to better coordinate the assistance needed.

A more technical aspect to the project is the way the internal fall data is handled. As the fall detection application is a context-aware software system, it would be nice to use a context middleware to handle the detected fall features. We put a lot of effort in integrating the IST-MUSIC framework for context-awareness\footnote{http://ist-music.berlios.de/site/} into our Android application. Unfortunately when it came to integrating the OSGi framework IST-MUSIC heavily relies on with the Android SDK, we were left without any specific documentation (IST-MUSIC was primarily built for JAVA ME and JAVA SE and ported to Android in the final stage). The java code implementing a complete fall ontology, the context models for each fall feature, a context plugin and a context listener specially written for the IST-MUSIC integration resides in a separate branch in our code repository\footnote{See https://github.com/krikis/FallDet/tree/contexts.}. Future development of the fall detection application should focus on further exploration of the porting of OSGi to Android and alternative solutions that use the available Android apis for context management.

A last point of evaluation is the mode the Android application runs in. Currently the fall detection application runs as a standard Android `activity', that is: it can be launched by the user when he or she desires it and is suspended when the user starts another application. In order to make the fall detection really pervasive, the application should be started up when the phone is turned on and it should keep running in the background while the user is using other applications. This can be implemented with an Android `service'. Future work should focus on migrating the applications functionality to such a service so that falls are always detected, no matter what the user is doing with the smartphone.



\section{Results}
To try to validate the thresholds described in chapter 2 a few test were performed, namely:
\begin{itemize}
\item Walking
\item Biking 
\item Running
\item Letting the phone fall from different distance into a bed  and stairs
\end{itemize}

\newpage
\subsection{Bed and Stairs}
To test if the thresholds were going to detect a fall correctly we decided to first try throwing the phone on the bed from different distances and throwing it from the stairs. The thresholds were -0.7 for the VVE and 2.8 for the RSS.
\image{im1.png}{Fall from 50cm, VVE -0.7 and RSS 2.8} 
\image{im2.png}{Fall from 80cm, VVE -0.7 and RSS 2.8}
\image{im3.png}{Fall from 120cm, VVE -0.7 and RSS 2.8}
\image{im4.png}{Fall from 200cm, VVE -0.7 and RSS 2.8}
\image{im5.png}{Fall from the stairs, VVE -0.7 and RSS 2.8}
As is shown in figure 1 to 5 no fall was detected no matter what distance was used. Carefully looking at the pictures we can see that the vertical velocity is always very close to the point where a fall should be detected. Because of that we decided to change the threshold value of the VVE from -0.7 to -0.4. To come up with this threshold we let the phone fall from 200cm and check at which point it would recognize a fall. The first time a fall was recognized was when the threshold was -0.4.
\image{im6.png}{Fall from 200cm, VVE -0.4 and RSS 2.8}
\image{im7.png}{Fall from stairs, VVE -0.4 and RSS 2.8}
As shown in the pictures from 6 to 7 the fall was recognized with the new threshold. Because of that it was decided to continue using those thresholds on the following tests.

\subsection{Walking}

In the walking test the phone was placed in the back pocket of a short. The phone was a bit loose, so it could move around a few centimeters. Also, the walking test was performed with the phone in the front pocket of a  jeans but no pictures are available of that particular test. 
\image{im8.png}{Walking test with the phone in the back pocket of a jeans. VVE -0.4 and RSS 2.8}
As shown in Figure 8 both the values of the RSS and VVE stay above the threshold as expected.

\subsection{Running}
To further test the ability of the phone to detect a fall we decided to test the phone while running. The results of that test were a bit mixed. In this test the phone was also placed in the back pocket of a short. Figure 9 shows a fall detected while running. We think this can happen because during the test I removed the phone too fast of the pocket. This resulted in the Posture file going over the threshold and detecting the fall. Figure 11 shows what happens while a person is running and wait a few seconds before taking the phone out of the pocket. Here we see that no fall was detected. Also in this picture we can see that the fall was detected in the beginning of the running and during the running no fall was detected. So this confirms our previous hypothesis that the fall detected in Figure 9 was caused by removing the phone too fast out of the pocket. Figure 10 shows another run with the phone in the back pocket. Here we can see no fall was detected. 
\image{im9.png}{Fall detected while running. VVE -0.4 and RSS 2.8}
\image{im10.png}{No fall detected while running. VVE -0.4 and RSS 2.8}
\image{im11.png}{No fall detected while running and waiting a few seconds to pick up the phone. VVE -0.4 and RSS 2.8}

\subsection{Biking}
The application was also tested while biking but no pictures are available. But no fall was detected while biking.

\subsection{Conclusion}

The application worked well in the performed tests. The falls that we would like to detect were detected while avoiding false falls. However there was an occasion where a fall was detected while running that wasn’t supposed to happen.  But that was caused by removing the phone too quickly from the pocket. This illustrates that the application can be tricked into detecting a fall where no fall exist. This still is countered up to a certain point by the dialog that is displayed to the user each time a fall is detected. The application has a downside however. While running the application the battery of the phone will be drained very fast. With the Sony Ericsson Xperia x10 the phone life was reduced to around 7 hours when the app was running. Further investigation must be conducted to evaluate if the code can be optimized further to reduce the amount of power drain. Also further investigation should be conducted to see if other Android phones suffer from the same effect as the Xperia.  



\end{document}
